\section{НОД и НОК}

%------------------------------------------------

\begin{frame}
    \center \Huge НОД и НОК 
\end{frame}

%------------------------------------------------

\begin{frame}
	\frametitle{НОД и НОК}

    \quad  $gcd(x,y) = \min\{a~|~x \vdots a~and~y \vdots a\}$

    \quad \textbf{НОД} - Наименьший общий делитесь. Простыми словами --- наименьшее число, на которое делятся оба числа.

    \quad  $lcm(x,y) = \min\{a~|~x \vdots a~and~y \vdots a\}$

    \quad \textbf{НОК} - Наибольшее общее кратное. Простыми словами --- наибольшее число, которое делиться на оба числа.
\end{frame}

%------------------------------------------------

\begin{frame}
	\frametitle{Нахождение НОД и НОК}
    
    \quad Как найти НОД и НОК вручную?

    \begin{enumerate}
        \item Разложим оба числа в на простые делители

        \item 
            \begin{itemize}
                \item Для нахождения НОК найдём объединений полученные множеств делителей
                \item Для нахождения НОД найдём пересечение полученные множеств делителей
            \end{itemize}

        \item Перемножим все элементы множества из шага 2
    \end{enumerate}

    \quad НОД и НОК связаны между собой формулой $\text{НОК}(x,y)=\dfrac{x*y}{\text{НОД}(x,y)}$
\end{frame}

%------------------------------------------------

\begin{frame}
	\frametitle{Нахождение НОД и НОК}
    
    \quad Посчитаем на примере чисел 132 и 270\\

    \begin{tabular}{c|ccc|c}
        132 & 2  & ~ & 270 & 2\\
        66  & 2  & ~ & 135 & 5 \\
        33  & 3  & ~ & 27  & 3 \\
        11  & 11 & ~ & 9   & 3 \\
        1   & ~  & ~ & 3   & 3 \\
        ~   & ~  & ~ & 1   & 
    \end{tabular}
    
    \quad Получили множества $\{2,2,3,11\}$, $\{\2,3,3,3,5\}$

    \quad $gcd(132,270) = 2*3 = 6$, $lcm(132,270) = 2*2*3*11*3*3*5=5940$

    \quad $lcm(132,270) = \dfrac{132*270}{gcd(132,270)} = \dfrac{35640}{6}=5940$

\end{frame}

%------------------------------------------------

\begin{frame}[fragile]
	\frametitle{Алгортитм Евклида}

\begin{cpp}
int gcd(int x, int y){
    if(y == 0){
        return x;
    }

    if (x < y){
        swap(x,y);
    }

    return gcd(y, x%y);
}
\end{cpp}

\quad Работает за O(log(min(x,y)))

\quad В стандартной библиотеке шаблонов уже реализована функция gcd(x,y);

\end{frame}

%------------------------------------------------

\begin{frame}
    \center \Huge Время контеста 
\end{frame}

%------------------------------------------------

