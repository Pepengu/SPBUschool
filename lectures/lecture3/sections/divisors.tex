\section{Делители}

%------------------------------------------------

\begin{frame}
    \center \Huge Делители 
\end{frame}

%------------------------------------------------

\begin{frame}
	\frametitle{Представление чилса}

    \quad Математическое определение $\forall x, y ~~\exists a, b: y > b \geq 0 \Rightarrow x = ay + b$. В этом определении $b$ --- остаток от деления числа. По этому для $x=-10, y=3$ коректным выражением является $-10 = -4 \times 3 + 2$

    \quad В с++ нет ограничения на знак остатка! При этом $a$ является минимальным по модулю. Таким образом $-10\%3$ вернёт результат $-1$. Чтобы это обойти приходится писать $(a\%b+b)\%b$


\end{frame}

%------------------------------------------------

\begin{frame}
	\frametitle{Делители}

\quad Если $b=0$, то говорят что $a$ и $y$ называют \textbf{делителями числа}. Пишется как $b\vdots a$ или $a|b$
    
    \quad Програмно проверку числа на делимость осуществить просто: $a\%b~==~0$ либо $!(a\%b)$

    \quadПоскольку $a<<2$ и $a/2$, то можно писать $!(a\text{\&}1)$, что является более быстрой проверкой на чётность. 

\end{frame}

%------------------------------------------------

\begin{frame}[fragile]
	\frametitle{Нахождение делителей числа. Тривиальный алгоритм}

    \begin{cpp}
int count_divisors(int n){
    int cnt = 0;
    for(int i = 1; i <= n; ++i){
        cnt += n%2==0;
    }
    return cnt;
}
    \end{cpp}

\end{frame}

%------------------------------------------------

\begin{frame}[fragile]
	\frametitle{Нахождение делителей числа. Более оптимальная версия}

    \quad В действительности смотреть на все n чисел не имеет смылса. Рассмотрим делители числа 24. Оно делиться на $1, 2, 3, 4, 6, 8, 12, 24$. Когда мы доходим до числа $2$, мы можем составить выражение $24=2*12$, то есть находя 2 мы уже можем сказать ещё один делитель. Рассмотрим все такие пары: $(1,24),(2,12),(3,8),(4,6)$. Как понять сколько таких пар? Если мы смотрим на наименьшее число в паре, то оно не может превосходить корень числа. Если число является полным квадратом, то корень можно рассматривать как пару равных чисел, например для $36$ это пара $(6,6)$. Такие корни называются \textbf{чётными корнями}.
    
\end{frame}

%------------------------------------------------

\begin{frame}
    \center \Huge Время контеста 
\end{frame}

%------------------------------------------------


\begin{frame}[fragile]
	\frametitle{Нахождение делителей числа. Алгоритм за O(\sqrt{n})}

    \begin{cpp}
vector<int> divisors(int n){
    vector<int> res;
    res.push_pack(1);
    for(int i = 2; i*i <= n; ++i){
        if(n%2==0){
            res.push_pack(i);
            if(i*i!=n){
                res.push_pack(n/i);
            }
        }
    }
    res.push_pack(n);
    return res;
}
    \end{cpp}

\end{frame}
