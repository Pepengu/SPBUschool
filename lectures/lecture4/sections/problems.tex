\section{Примеры}

\begin{frame}
    \quadЗадача: Есть полоска $1$×$n$. Кузнечик стоит на первой клетке, он может прыгать вперед на 1, 2 или 3 клетки. Сколько есть способов добраться от начальной клетки до последней?
\end{frame}

\begin{frame}
    Подзадача --- основная задача для полоски 1xk.\\
    Как получить ответ на бОльшую задачу из меньших?
\end{frame}

\begin{frame}
    Усложним задачу --- на некоторые клетки прыгать нельзя, там препятствие.\\
    Как изменится решение?
\end{frame}

\begin{frame}[fragile]
    \begin{cpp}
dp[0] = 1
for (int i = 1; i < n; i++) {
    for (int j = i - 1; j > max(i - k, 0); j--) {
        dp[i] += dp[j] * a[j];
    }
}

for (int i = 1; i < n; i++) {
    for (int j = i + 1; j <= min(i + k, 0); j++) {
        dp[i + j] += dp[i] * a[i];
    }
}
    \end{cpp}
\end{frame}

\begin{frame}
    Другая задача --- дано число n. Сколько существует способов набрать число n бросками кубиков к6?
\end{frame}

\begin{frame}
    \quad Определимся с подзадачами --- очевидно, задача для $k$($k \le n$).
    Но какой переход?
\end{frame}

\begin{frame}
    И последняя задача в презентации:\\
    Существует n номиналов монет. У вас их неограниченное количество. Нужно найти минимальное количество монет, необходимое чтобы набрать сумму в x.
\end{frame}

\begin{frame}
    Подзадача --- набрать x-1, x-2\dots монет
\end{frame}

\begin{frame}
    Переход --- если у нас есть монета номиналом k и мы можем набрать сумму x-k используя m монет, то x можем набрать за m+1 монету
\end{frame}