\section{Вступление}

\begin{frame}
    Представим задачу --- посчитать n-ое число Фибоначчи\\
    Самый простой способ --- сначала посчитать n-1 и n-2 числа\\
    \quadОднако, если считать ``втупую``, то сложность будет как минимум $O(2^(n/2))$
\end{frame}

\begin{frame}
    Основная идея(разбить сложную, большую задачу на задачи попроще и поменьше) --- правильная\\
    Однако надо не допускать экспоненциального роста количества подзадач\\
    \quad Один раз посчитали значение $f_n$ --- запомнили его и больше не считаем
\end{frame}

\begin{frame}
    Основная идея динамического программирования --- сведение большой задачи к задаче поменьше\\
    \quad Обычно, свести задачу для числа $n$ к задаче для чисел меньших $n$\\
    \quad Очень важный вопрос для решения динамики --- что будет являться ответом на подзадачу?\\
    \quad Не менее важный вопрос --- если у нас есть решение для подзадачи, как получить решение задачи?
\end{frame}

