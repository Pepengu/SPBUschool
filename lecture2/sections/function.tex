\section{Бинпоиск по функции}

\begin{frame}
    \frametitle{Другой взгляд}
    Бинпоиском можно искать значение нестрого монотонной функции\\
    \small Массив --- в целом тоже функция array(i), возвращающая элемент под индексом i\\
    \small Обычно бинпоиск на самом деле ищет первое большее/меньшее/больше либо равное/меньше либо равное 
\end{frame}

\begin{frame}
    \frametitle{Другой взгляд}
    Бинпоиском можно искать значение нестрого монотонной функции\\
    \small Массив --- в целом тоже функция array(i), возвращающая элемент под индексом i 
\end{frame}

\begin{frame}
    \frametitle{Пример задачи}
    Есть n прямоугольников одинакового размера: w в ширину и h в длину. Требуется найти квадрат минимального размера, в который можно упаковать данные прямоугольники. Прямоугольники при этом нельзя поворачивать.
\end{frame}

\begin{frame}
    Придумать как это решить математически или конструктивно --- достаточно сложно\\
\end{frame}

\begin{frame}
    Придумать как это решить математически или конструктивно --- достаточно сложно\\
    Однако проверить влезут ли они в квадрат со стороной x --- легко
\end{frame}

\begin{frame}
    Придумать как это решить математически или конструктивно --- достаточно сложно\\
    Однако проверить влезут ли они в квадрат со стороной x --- легко\\
    Решение --- бинпоиск по стороне квадрата
\end{frame}

\begin{frame}[fragile]
    \frametitle{Код решения}
    \begin{cpp}
long long l = 0, r = 1e18;
while (l + 1 < r) {
    long long m = (l + r) / 2;
    long long s = (m / w) * (m / h);
	if (s > n) {
		r = m;
	} else l = m;
}
return l;
    \end{cpp}
\end{frame}
